\newcommand{\ClassPath}{.}
\documentclass{\ClassPath/viu-tfm-template}

% Color principal de la VIU. Si lo modificas, puedes aprovechar
% la plantilla para otras cosas.
\definecolor{maincolor}{HTML}{f25416}

%--------------------------------------------------------------------------
% Definiciones necesarias Modifica con tus datos
%--------------------------------------------------------------------------
\def\nombre{Apellido1 Apellido2, Nombre}
\def\dni{12345678-A}
\def\titulo{Título del TFM}
\def\titulacion{Máster Universitario en Desarrollo de Aplicaciones y Servicios Web}
\def\curso{2022-2023}

%Los siguientes son opcionales: si no se ponen, la portada cambia un poco. Ideal para escribir artículos/trabajos cortos
\def\dirige{Nombre del director/a}
\def\codirige{Nombre del codirector/a}
\def\convocatoria{Primera}
% No pongas nombre de la asignatura si pones algo en “\dirige”, porque no tiene sentido.
\def\asignatura{}

\RequirePackage{blindtext} % se puede borrar al escribir el TFM

% importar fichero de Bibliografía
\addbibresource{bibliography.bib}

\begin{document}
    \coverpage


    %--------------------------------------------------------------------------
    % Abstract
    %--------------------------------------------------------------------------

    % Creo un “abstract” propio, porque la plantilla “book” no la tiene, y cambiar a “report” no aporta nada nuevo. Aparte, el comando “abstract” original hace salto de página.

    \vspace*{\fill}
    \begin{center}
        \textbf{Resumen}
    \end{center}

    Lorem ipsum dolor sit amet, consectetur adipiscing elit. Vestibulum pretium libero non odio tincidunt semper. Vivamus sollicitudin egestas mattis. Sed vitae risus vel ex tincidunt molestie nec vel leo. Vestibulum ante ipsum primis in faucibus orci luctus et ultrices posuere cubilia Curae; Maecenas quis massa tincidunt...

    \keywords{one, two, three, four}

    \vspace*{\fill}
    \vspace*{\fill}
    \vspace*{\fill}

    \pagebreak

    %--------------------------------------------------------------------------
    % end of Abstract
    %--------------------------------------------------------------------------

    % opcional en artículos/trabajos/actividades cortos
    \tableofcontents

    \chapter{Ejemplo de cita para bibliografía}
    Tal como aparece en \textcite{einstein}, ...

    Distintos tipos de citas:

    Como dice \textcite{Xanthopoulos} con textcite

    \cite{Xanthopoulos}  con cite

    \parencite{Xanthopoulos} con parencite

    \citeauthor{Xanthopoulos} con citeauthor

    \chapter{Ejemplo de código fuente}
    Un ejemplo de código de terminal GNU/Linux:

\begin{mycode}{Config file}{console}{}
[root@localhost ~]# vi /etc/systemd/journald.conf
\end{mycode}


    Un ejemplo de código en Python:

\begin{mycode}{Hello World}{python}{}
#!/usr/bin/env python

# Defining main function
def main():
    print("Hello World!")

# Using the special variable: __name__
if __name__=="__main__":
    main()
\end{mycode}

Para genera estos códigos, se tiene que utilizar el siguiente código en \LaTeX:
\begin{mycode}{Ejemplo}{latex}{}
\begin{mycode}{Título}{lenguaje}{\normalsize}
\end{mycode}

Y luego se cierra el bloque. Lo mejor es mirar el código fuente de este fichero.


    \chapter{Ejemplo de requisitos}
    Ejemplo de requisitos en ingeniería de software:

    \begin{requisitostbl}{X[-1]X[1]X[1]X[1]X[1]}
        ID & Tipo & Categoría & Prioridad &  Dependencias \\
        1  & No Funcional & Producto & Must & 1  \\

        Título del requisito \\

        \textbf{Descripción}:

            Bla bla bla...  \\

        \textbf{Razón}:

            Bla bla bla... \\
    \end{requisitostbl}


    % ejemplo de texto, del paquete “blindtext”. Borrar al usar la plantilla.
    \Blinddocument

    % sólo si se tiene bibliografía
    \printbibliography[title={Referencias bibliográficas},heading=bibintoc]
\end{document}