\newcommand{\ClassPath}{.}
\documentclass{\ClassPath/viu-tfm-template}

% Color principal de la VIU. Si lo modificas, puedes aprovechar
% la plantilla para otras cosas.
\definecolor{maincolor}{HTML}{f25416}

%--------------------------------------------------------------------------
% Definiciones necesarias Modifica con tus datos
%--------------------------------------------------------------------------
\def\nombre{Apellido1 Apellido2, Nombre}
\def\dni{12345678-A}
\def\titulo{Título del TFM}
\def\titulacion{Máster Universitario en Desarrollo de Aplicaciones y Servicios Web}
\def\curso{2022-2023}

%Los siguientes son opcionales: si no se ponen, la portada cambia un poco. Ideal para escribir artículos/trabajos cortos
\def\dirige{Nombre del director/a}
\def\convocatoria{Primera}
% No pongas nombre de la asignatura si pones algo en “\dirige”, porque no tiene sentido.
\def\asignatura{}

\RequirePackage{blindtext} % se puede borrar al escribir el TFM

% importar fichero de Bibliografía
\addbibresource{bibliography.bib}

\begin{document}
    \coverpage

    % opcional en artículos/trabajos/actividades cortos
    \tableofcontents

    \chapter{Ejemplo de cita para bibliografía}
    Tal como aparece en \textcite{einstein}, ...

    Distintos tipos de citas:

    Como dice \textcite{Xanthopoulos} con textcite

    \cite{Xanthopoulos}  con cite

    \parencite{Xanthopoulos} con parencite

    \citeauthor{Xanthopoulos} con citeauthor

    \chapter{Ejemplo de requisitos}
    Ejemplo de requisitos en ingeniería de software:

    \begin{requisitostbl}{X[-1]X[1]X[1]X[1]X[1]}
        ID & Tipo & Categoría & Prioridad &  Dependencias \\
        1  & No Funcional & Producto & Must & 1  \\

        Título del requisito \\

        \textbf{Descripción}:

            Bla bla bla...  \\

        \textbf{Razón}:

            Bla bla bla... \\
    \end{requisitostbl}


    % ejemplo de texto, del paquete “blindtext”. Borrar al usar la plantilla.
    \Blinddocument

    % sólo si se tiene bibliografía
    \printbibliography[title={Referencias bibliográficas},heading=bibintoc]
\end{document}